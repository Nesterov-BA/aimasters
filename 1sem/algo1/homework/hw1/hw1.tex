\documentclass{article}
\usepackage[T2A]{fontenc}
\usepackage[utf8]{inputenc}
\usepackage[english, russian]{babel}

% Set page size and margins
% Replace `letterpaper' with`a4paper' for UK/EU standard size
\usepackage[a4paper,top=2cm,bottom=2cm,left=2cm,right=2cm,marginparwidth=1.75cm]{geometry}

\usepackage{amsmath}
\usepackage{graphicx}
\usepackage[colorlinks=true, allcolors=blue]{hyperref}
\usepackage{amsfonts}
\usepackage{amssymb}
\usepackage{amsthm}
% \usepackage[left=1cm,right=1cm,top=1cm,bottom=1cm]{geometry}
\usepackage{hyperref}
\usepackage{seqsplit}
\usepackage[dvipsnames]{xcolor}
\usepackage{enumitem}
\usepackage{algorithm}
\usepackage{algpseudocode}
\usepackage{algorithmicx}
\usepackage{mathalfa}
\usepackage{mathrsfs}
\usepackage{dsfont}
\usepackage{caption,subcaption}
\usepackage{wrapfig}
\usepackage[stable]{footmisc}
\usepackage{indentfirst}
\usepackage{rotating}
\usepackage{pdflscape}

\usepackage{MnSymbol,wasysym}
\newtheorem{theorem}{Теорема}[section]
\newtheorem{lemma}{Лемма}[section]
\newtheorem{corollary}{Следствие}[section]
\newtheorem{proposition}{Утверждение}[section]
\newtheorem{theorem*}{Теорема}
\newtheorem{lemma*}{Лемма}
\newtheorem{corollary*}{Следствие}
\newtheorem{proposition*}{Утверждение}
\newtheorem{definition}{Определение}[section]
\newtheorem{remark}{Замечание}[section]
\newtheorem{example}{Пример}[section]
\newtheorem{definition*}{Определение}
\newtheorem{remark*}{Замечание}
\newtheorem{example*}{Пример}


\begin{document}

\begin{center}
  \large{\textbf{Нестеров Борис Аркадьевич
    }}

  \Large {Задание 1. Асимптотические сложности.}
\end{center}

\bigskip

\textbf{1} Известно, что $f(n) = O(n^2),\ g(n) = \Omega(1),\ g(n) = O(n)$. Положим $$h(n) = \cfrac{f(n)}{g(n)}.$$

1. Возможно ли, что \textbf{а)} $h(n) = \Theta(n\log n)$; \textbf{б)} $h(n) = \Theta(n^3)$?

Из условия задачи следует, что $f(n)$ растет не быстрее $Cn^{2}$, $g(n)$
отделена от $0$ и растет не быстрее $Cn$.  Далее будем пользоваться эквивалентым
определением О большого, а именно:
\begin{center}
  Далее предполагаем, что функции не равны $0$ ни при каких $n$, кроме конечного количеста $n$.
  $f(n) = O(g(n))$, если $\frac{f(n)}{g(n)}$ --- ограниченная
  функция. В частности, очевидно, $f(n) = O(f{n})$. Отслюда также вытекает, что если $f(n) = O(g(n))$, то $f(n)h(n) = O(g(n)h(n))$.
\end{center}
\textbf{а)} Возьмем $f(n) = n^{2}, g(n) = \frac{n}{\log n}$, $n^{2} = O(n^{2})$ -- очевидно,
$\frac{\frac{n}{\log n}}{n} = \frac{1}{\log n}$ -- ограниченная функция. Итак, $h(n) = \frac{f(n)}{g(n)} = n \log n$
\textbf{Ответ:} да.

\textbf{б)} Если $h(n) = \Theta(n^{3})$, то это означает, что существует константа $C_{1}$ такая, что $\frac{f(n)}{g(n)} > C_{1}n^{3}$. Это неравенсто эквивалентно $g(n) < \frac{f(n)}{C_{1}n^{3}} = O(\frac 1 n)$. Это означает, что $g(n) = O(\frac 1 n)$.
Это противоречит отделимости $g(n)$ от $0$.
\textbf{Ответ:} нет.

2. Приведите наилучшие (из возможных) верхние и нижние оценки на функцию $h(n)$ и приведите пример функций $f(n)$ и $g(n)$ для которых ваши оценки на $h(n)$ достигаются.

$g(n)$ ограничено снизу константой, $f(n)$ ограничено сверху $Cn^{2}$
\medskip

Перед решением следующих задач, докажем вспомогательное утверждение.
\begin{theorem*}
  Пусть задана положительная монотонная функция $f(x): \mathbb{R}^{+}\rightarrow \mathbb{R}^{+}$.
  Обозначим $\sum_{k = 1}^{n}f(k) = F(n)$.
  Для неубывающих $f$ получим $G(n) + C \le F(n) \le G(n+1)$, а для невозрастающих --
  $G(n + 1) \le F(n) \le G(n)+C$
  где $G(x) = \int_{1}^{x} f(t)dt$
\end{theorem*}
\begin{proof}
  Рассмотрим случай монотонно возрастающей функции $f(x)$, случай убывающей полностью аналогичен.
  На полуинтервале $[k,k+1), f([x]) = f(k) \le f(x)$. Получаем
  \[\sum_{1}^{n}f(k) = \int_{1}^{n+1}f([x])dx \le \int_{1}^{n+1}f(x)dx = G(n+1)\]
  \[\sum_{1}^{n}f(k) = \int_{1}^{n+1}f([x])dx \ge \int_{1}^{n+1}f(x-1)dx = G(n) + \int_{0}^{1}f(x)dx\]
  Для убывающих функций, например $\frac 1 n$, может возникнуть проблема с определением их интеграла в $0$.
  В таком случае будем рассматривать сумму начиная с $2$, что не изменит ассимптотику суммы.
\end{proof}
\begin{corollary}
  Для функций $f$ вида $n^{k} \log^{l} n, k\ge -1$ и $e^{an}$ из теоремы следует $F(n) = \Theta(G(n))$.
\end{corollary}
\textbf{2} Найдите $\Theta$-асимптотику $\sum\limits_{i=1}^n \sqrt{i^3+2i+5}$.
\[f(k) = \sqrt{k^{3} + 2k + 5}\]
Начиная с некоторого $k_{0}$, а именно с наибольшего корня ур-я $3k^{3} = 2k + 5$,
\[k^{\frac 3 2} < f(k) < 2k^{\frac 3 2}.\]
Это означает, что
\[\sum_{1}^{n} k^{\frac 3 2} < \sum_{1}^{n}f(k) < \sum_{1}^{n}2k^{\frac 3 2}.\]
\[\int x^{\frac 3 2}dx = x^{\frac 5 2} + C.\]
Получааем, что в неравенстве слева и справа стоят функции $\Theta(n^{\frac 5 2})$
$\sum\limits_{i=1}^n \sqrt{i^3+2i+5} = \Theta(n^{\frac 5 2})$.

\medskip

\textbf{3} Докажите, что асимптотика $\sum^n_{i=1} i^{\alpha}
= \Theta(n^{1+\alpha})$, если
$\alpha > 0$.

\[\int x^{\alpha}dx = x^{\alpha+1}+C\]. Далее см. Следствие 1.
\medskip

\textbf{4} Найдите $\Theta$-асимптотику функции
$g(n) = \sum ^n_{k=1} \frac{1}{k};$

\[\int_{1}^{n} \frac{1}{x}dx = \ln n\]. Тогда, по следствию 1,
\[g(n) = \Theta(\ln n)\]
\medskip

\end{document}
