\documentclass{article}
\usepackage{myrussian}
\begin{document}

\begin{center}
  \Large {Домашнее задание 1.}
\end{center}
\bigskip
\section{Выпуклые множества}
\textbf{1.}
а) Выпуклость \( \Rightarrow \) Пересечение с любой прямой выпукло.
\\
Очевидно, т.к. прямая  --- тоже выпуклое множество.
\\
б) Пересечение с любой прямой выпукло \( \Rightarrow \) Выпуклость.
\\
Пусть наше множество --- \( X \). Если оно состоит из одной точки,
оно выпукло. Далее, в нем лежат две различные точки \( x_1,x_2 \).
Проведем через эти точки прямую. По условию, пересечение прямой с \(
X \) выпукло. Это означает, что в пересечении лежат все точки вида \(
tx_1 + (1-t)x_2 \), \( t \in [0,1] \). Значит эти точки лежат и в \(
X \). Получили, что вместе с каждой парой точек во множестве \( X \)
лежит отрезок, их соединяющий. Это определение выпуклости.
\medskip

\textbf{2.}
\medskip

\textbf{3.}
\medskip

\textbf{4.}
Докажем, что множество \( \big\{x\in \mathbb{R}^n \mid \| x-x_0 \|_2 \le \|
x-y_0 \|_2\big\} \) выпукло. Тогда исходное будет выпукло, как
пересечение выпуклых по \( S \).
\section{Двойственные конусы}
\textbf{2.} Далее \( X = \{x\in \mathbb{R}^n \mid x_1 \ge
x_2\ge\ldots\ge x_n\ge 0\} \). Замкнутость и выпуклость следуют из
того, что это множество является пересечением полупространств вида \(
x_i \ge x_{i+1} \) и полупространства \( x_n \ge 0 \). Каждое из них
является выпуклым и замкнутым, значит и пересечение выпукло и
замкнуто. Рассмотрим \( x^1,x^2 \in X \), \( x^i = (x^i_1,\ldots ,
x^i_n) \) и возьмем линейную комбинацию \( t_1x^1 + t_2x^2 \), \(
t_1,t_2 \ge 0 \). Очевидно, что если \( x^1_i \ge x^1_{i+1} \) и \(
x^2_i \ge x^2_{i+1} \), то \(t_1x^1_i + t_2 x^2_i \ge t_1x^1_{i+1} + t_2
x^2_{i+1} \). Также \(  t_1x^1_n + t_2 x^2_n \ge 0\). Значит \(
t_1x^1+t_2x^2 \in X \). \( X \) равен своей конической оболочке, т.е.
он --- конус.
\end{document}
