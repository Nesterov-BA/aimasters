\documentclass{article}
\usepackage[T2A]{fontenc}
\usepackage[utf8]{inputenc}
\usepackage[english, russian]{babel}

% Set page size and margins
% Replace `letterpaper' with`a4paper' for UK/EU standard size
\usepackage[a4paper,top=2cm,bottom=2cm,left=2cm,right=2cm,marginparwidth=1.75cm]{geometry}

\usepackage{amsmath}
\usepackage{graphicx}
\usepackage[colorlinks=true, allcolors=blue]{hyperref}
\usepackage{amsfonts}
\usepackage{amssymb}
% \usepackage[left=1cm,right=1cm,top=1cm,bottom=1cm]{geometry}
\usepackage{hyperref}
\usepackage{seqsplit}
\usepackage[dvipsnames]{xcolor}
\usepackage{enumitem}
\usepackage{algorithm}
\usepackage{algpseudocode}
\usepackage{algorithmicx}
\usepackage{mathalfa}
\usepackage{mathrsfs}
\usepackage{dsfont}
\usepackage{caption,subcaption}
\usepackage{wrapfig}
\usepackage[stable]{footmisc}
\usepackage{indentfirst}
\usepackage{rotating}
\usepackage{pdflscape}

% \usepackage{minted}

\usepackage{MnSymbol,wasysym}

\begin{document}

\begin{center}
  \Large {Задание 4. Рекурренты.}
\end{center}

\textbf{1} Обозначим $v_n$ --- двумерный вектор-столбец равный
$
\begin{pmatrix}
  F_{n}\\
  F_{n-1}
\end{pmatrix}.
$
Из соотношения  $F_{k+2} = F_{k+1} + F_k$ необходимо получить матрицу
перехода от $v_n$ к $v_{n+1}$, т. е.
\[
v_{n+1} = Av_n\]
\[
  \begin{pmatrix}
    F_{n} + F_{n-1}\\
    F_{n}
  \end{pmatrix} = A
  \begin{pmatrix}
    F_{n}\\
    F_{n-1}
  \end{pmatrix}
\]
Получаем, что $A =
\begin{pmatrix}
  1 &1\\
  1 &0
\end{pmatrix} =
\begin{pmatrix}
  F_2 &F_1\\
  F_1 &F_0
\end{pmatrix} $. Столбцы матрицы $A$ равны $v_2, v_1$, из чего получаем формулу:
\[A^n =
  \begin{pmatrix}
    F_{n+1} &F_n\\
    F_{n} &F_{n-1}
\end{pmatrix} .\]
Так, соотношения верны и для отрицаетльных $n$, для нахождения
$F_{-n}$ достаточно посчитать \[A^{-n} = (A^{-1})^n =
  \begin{pmatrix}
    0 & 1\\
    1 & -1
\end{pmatrix}^n.\] Искомое число будет лежать в левом нижнем (или
правом верхнем) углу матрицы.
\medskip

\textbf{2} Найдите асимптотическую оценку функции $T(n)$. Примените
мастер-теорему в тех случаях, когда ее можно использовать, \textbf{и
посчитайте асимптотику иначе, когда нельзя}. Варианты есть следующие.
Можно выписать рекурренту в виде суммы и найти, чему она равна. Можно
подставить рекурренту саму в себя и посмотреть, что получается. Можно
обратиться к литературе (учебник Кормена, учебник Дасгупты)

\begin{enumerate}
  \item $T(n) = 25 T(\frac{n}{5}) + n^2$
  \item $T(n) = 16 T(\frac{n}{2}) + n^3$
  \item $T(n) = 9 T(\frac{n}{3}) + n^3$
  \item $T(n) = T(n-1) + 3n$
  \item $T(n) = T(\frac{n}{4}) + T(\frac{3n}{4}) + n$
  \item $T(n) = T(\frac{n}{2}) + T(\frac{2n}{3}) + n^2$
  \item $T(n) = T(n - 1) + n^2$
  \item $T(n) = 4 T(\frac{n}{16}) + \sqrt{n}$
  \item $T(n) = 9 T(\frac{n}{3}) + n^3$
  \item $T(n) = 9 T(\frac{n}{3}) + n$
\end{enumerate}
\textbf{Решение}
\begin{enumerate}
  \item \( \log_b a = 2 \), \( c=2, k = 0
    \) \( \Rightarrow \) \( T(n) = \Theta (n^2 \log n) \)
  \item $\log_{b}a = 4, c = 3 \Rightarrow T(n) = \Theta(n^{4})$
  \item $\log_{b}a = 2, c = 3 \Rightarrow T(n) = \Theta(n^{3})$ \label{item 3}
  \item \( T(n) \) - сумма линейных слагаемых, значит \( T(n) = \Theta (n^2) \)
  \item ---
  \item ---
  \item \( T(n) \) --- сумма квадратичных слагаемых, значит \( T(n) =
    \Theta (n^3) \)
  \item \( \log_b a =\frac{1}{2} \), \( c=\frac 1 2, k = 0
    \) \( \Rightarrow \) \( T(n) = \Theta (\sqrt n \log n) \)
  \item Это пункт \ref{item 3}
  \item $\log_{b}a = 2, c = 1 \Rightarrow T(n) = \Theta(n^{2})$
\end{enumerate}

\end{document}
