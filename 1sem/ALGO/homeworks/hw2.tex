\documentclass{article}
\usepackage[T2A]{fontenc}
\usepackage[utf8]{inputenc}
\usepackage[english, russian]{babel}

% Set page size and margins
% Replace `letterpaper' with`a4paper' for UK/EU standard size
\usepackage[a4paper,top=2cm,bottom=2cm,left=2cm,right=2cm,marginparwidth=1.75cm]{geometry}

\usepackage{amsmath}
\usepackage{graphicx}
\usepackage[colorlinks=true, allcolors=blue]{hyperref}
\usepackage{amsfonts}
\usepackage{amssymb}
% \usepackage[left=1cm,right=1cm,top=1cm,bottom=1cm]{geometry}
\usepackage{hyperref}
\usepackage{seqsplit}
\usepackage[dvipsnames]{xcolor}
\usepackage{enumitem}
\usepackage{algorithm}
\usepackage{algpseudocode}
\usepackage{algorithmicx}
\usepackage{mathalfa}
\usepackage{mathrsfs}
\usepackage{dsfont}
\usepackage{caption,subcaption}
\usepackage{wrapfig}
\usepackage[stable]{footmisc}
\usepackage{indentfirst}
\usepackage{rotating}
\usepackage{pdflscape}

\usepackage{minted}

\usepackage{MnSymbol,wasysym}

\begin{document}

\begin{center}
  \Large {Задание 2. Нижние оценки и числа Фибоначчи.}
\end{center}

\bigskip

\textbf{1.}
Воспользуемся биномом Ньютона: \( (a+b)^n =
\sum\limits_{k=0}^n{n\choose k} a^k b^{n-k} \). Видно, что если
подставить \( a = b = 1 \), получим \( f(n) \). Отсюда \( f(n) = 2^n
= \Theta (2^n) \).

\textbf{Ответ: }\( \Theta (2^n) \).
\medskip

\textbf{2.} \textbf{Алгоритм:} На каждом шаге делим кучу пополам,
если монет нечетное
количество, оставляем одну. Взвешиваем кучи, повторяем шаг для более
легкой. Если кучи равны по весу, значит монет было нечетное количество, и та,
которую оставили лежать --- фальшивая. Псевдокод:
\begin{algorithmic}[1]
  \Function{FindFalseCoin}{coins}
  \If{\(\#coins == 1\)}
  \State \Return coins
  \ElsIf{\(\#coins == 2n\)}
  \State \( \# coinsLeft = n, \#coinsRight = n \)
  \If{\( \text{mass}(coinsLeft) < \text{mass}(coinsRight) \)}
  \State \Call{FindFalseCoin}{coinsLeft}
  \Else
  \State \Call{FindFalseCoin}{coinsRight}
  \EndIf
  \ElsIf{\(\#coins == 2n+1\)}
  \State \( \# coinsLeft = n, \#coinsRight = n, singleCoin \)
  \If{\( \text{mass}(coinsLeft) < \text{mass}(coinsRight) \)}
  \State \Call{FindFalseCoin}{coinsLeft}
  \ElsIf{\( \text{mass}(coinsLeft) > \text{mass}(coinsRight) \)}
  \State \Call{FindFalseCoin}{coinsRight}
  \Else
  \State \Return singleCoin
  \EndIf
  \EndIf  \EndFunction
\end{algorithmic}
\textbf{Корректность: }На каждом шаге алгоритм применяется к той
куче, в которой лежит фальшивая монета. После каждого шага размер
кучи строго уменьшается: \( 2n \rightarrow n, 2n+1 \rightarrow n \),
если не происходит преждевременной остановки в случае нечетного числа
монет. Значит, за конечное число шагов алгоритм будет применен к
куче, состоящей из одной монеты и завершится, выдав фальшивую.
\\
\textbf{Асимптотика: } За один шаг размер кучи сокращается вдвое.
Пусть \( 2^{k-1} <\#\text{Монет} = n \le 2^k \). Отсюда \( k-1<\log n \le k \).
Обозначим \( T(n) \) --- количество шагов алгоритма для \( n \)
монет. Тогда \( k-1=T(2^{k-1})\le T(n) \le T(2^k) = k \). Т. е. \(
T(n) = \Theta(\log n) \).

\medskip
\textbf{3. }
Докажем, что задача нахождения фальшивой монеты имеет асимптотику \(
\log n \).
\\
Предположим, что дано \( 3^k \) монет, и мы пронумеруем их от
\( 0 \) до \( 22\ldots2 \) в троичной системе. Тогда для нахождения
фальшивой монеты необходимо узнать ее номер. Чтобы узнать одну цифру
номера, необходимо определить \( \frac 2 3 \) монет, как настоящие.
Для этого надо разделить монеты на 3 кучи и взвесить две. Заметим,
что больше одной цифры узнать за взвешивание нельзя, так как если в
каждой куче, которые взвешиваем будет более трети монет, то можем
опознать одну из них и оставшуюся, т. е. менее \( \frac 2 3 \), а
если взвешивать кучки менее трети монет, то в случае их равенства,
как настоящие мы определим менее \( \frac 2 3 \).
\\
Итак, каждое взвешивание дает не более одной цифры, всего цифр \( k
\), т. е. нужно \( k \)\ взвешиваний. Остается заметить, что если
количество монет находится между \( 3^{k-1} \) и \( 3^{k} \), то цифр
для записи нужно \( k = \lceil \log_3 n\rceil \).

\textbf{4.} \( 13 = 1101_2. \)
\begin{enumerate}
  \item\( b=1 \): \( \left(7^2\right)\cdot 7 \equiv 343 \equiv 343 - 334 \equiv
    9 \mod 167, \)
  \item\( b = 0\): \( \left(7^3\right)^2 \equiv 9^2 \equiv 81 \mod 167, \)
  \item\( b = 1\): \( \left(7^6\right)^2 \cdot 7 \equiv 81^2 \cdot 7
    \equiv 48\cdot 7 \equiv 2 \mod 167 \).
\end{enumerate}

\textbf{Ответ: }2.
\medskip
\end{document}
