\documentclass{article}
\usepackage[T2A]{fontenc}
\usepackage[utf8]{inputenc}
\usepackage[english, russian]{babel}

% Set page size and margins
% Replace `letterpaper' with`a4paper' for UK/EU standard size
\usepackage[a4paper,top=2cm,bottom=2cm,left=2cm,right=2cm,marginparwidth=1.75cm]{geometry}

\usepackage{amsmath}
\usepackage{graphicx}
\usepackage[colorlinks=true, allcolors=blue]{hyperref}
\usepackage{amsfonts}
\usepackage{amssymb}
% \usepackage[left=1cm,right=1cm,top=1cm,bottom=1cm]{geometry}
\usepackage{hyperref}
\usepackage{seqsplit}
\usepackage[dvipsnames]{xcolor}
\usepackage{enumitem}
\usepackage{algorithm}
\usepackage{algpseudocode}
\usepackage{algorithmicx}
\usepackage{mathalfa}
\usepackage{mathrsfs}
\usepackage{caption,subcaption}
\usepackage{dsfont}
\usepackage{wrapfig}
\usepackage[stable]{footmisc}
\usepackage{indentfirst}
\usepackage{rotating}
\usepackage{pdflscape}

\usepackage{minted}

\usepackage{MnSymbol,wasysym}

\def\ig#1#2#3#4{
  \begin{figure}[!ht]
    \begin{center}%
      \includegraphics[height=#2\textheight]{#1.eps}\caption{#4}\label{#3}%
    \end{center}
\end{figure}}

\newcommand{\floor}[1]{\left\lfloor #1 \right\rfloor}
\newcommand{\ceil}[1]{\left\lceil #1 \right\rceil}

\begin{document}

\begin{center}
  \Large {Задание 5. Битовые операции, модульная арифметика, немного реккурент.}
\end{center}
\bigskip

\textbf{1}
\medskip

\textbf{2}
\medskip

\textbf{3} Хотим найти \( a \cdot b \), где \( a \ge b \).
Надо вычислить \( c = a+b, d = a-b \). Это линейные операции сложения.
Далее вычисляем \( c^2, d^2 \) --- линейные по условию. Затем
вычисляем \( c^2 - d^2 \) при возведении в квадрат длина числа
умножается не более чем на 2, т.е. сложение снова линейно по изначальной длине.
Заметим, что \( c^2 - d^2 = (a+b)^2 - (a-b)^2 = 4ab \). Итак,
остается сделать поделить на \( 4 \), т.е. сделать два битовых сдвига.
\medskip

\textbf{4} Положим \(d = \text{НОД}(a,b)\) Так как \( a \vdots d\) и
\( b \vdots d \), \( a-b \vdots d \). Отсюда \( \text{НОД}(a-b,b)\ge
d \). Пусть теперь \( a-b, b \vdots d_1 > d \). Тогда и \( a\vdots d_1 \)
--- противоречие с тем, что \( d_1 \) --- НОД.
\medskip

\textbf{5} По предыдущей задаче, если вычесть из большего числа
меньшее, их НОД не изменится. То есть, после каждого вычитания НОД
всех чисел на доске не изменится.  Пусть
изначально на доске \( N \) чисел и их НОД равен \( d \).
Значит минимальная сумма всех чисел равна \( N \cdot d \). При каждом
вычитании по условию может получиться только положительное число,
значит после каждого вычитания на доске все числа положительны.
Значит каждое вычитание строго уменьшает сумму чисел на доске. Если
на доске осталось \( N \) одинаковых чисел меньше или больше \( d \),
то их НОД будет соответственно меньше или больше \( d \), что
невозможно. Итак сумма чисел убывает, значит процесс завершится, и
могут остаться только числа равные НОД изначально написанных.
\medskip

\textbf{6} Случаи \( \alpha = t \) и \( \alpha = 1-t \) идентичны,
поэтому будем рассматривать только \( \alpha \in
\left(0,\frac{1}{2}\right] \). Тогда \( \alpha \le 1- \alpha  \).
На каждом уровне рекурсии количество элементов суммируется к \( n \).
Поэтому на каждом уровне \( \Theta (n) \) операций. Оценим высоту
дерева рекурсии. Так как \( \alpha \le 1- \alpha \), \( \log_ \alpha
n \ge \log _{1 - \alpha }n \). В итоге получаем
\[
  \Theta (n)\log_{1-\alpha} n \le T(n) \le \Theta (n) \log_\alpha n
\]
\textbf{Ответ:} \( T(n) = \Theta (n\log n). \)
\medskip

\textbf{7} Условие записывается как \( T(n) = n \cdot T(\frac n 2) +
\Theta (n) \).
Высота дерева рекурсии равна \( \log_2 n \). На первом уровне \(
\Theta (n) \), на втором \( n \) задач по \( \Theta (\frac n 2) \),
на третьем \( n \cdot \frac n 2 \) на \( \Theta (\frac{n}{4}) \) и
т.д. На \( i \)-м уровне \( n \cdot \frac{n}{2} \cdot \dots \cdot
\frac{n}{2^{i-1}} = \frac{n^{i}}{2^{\frac{i(i-1)}2}}\). Итак,
асимптотика равна функции (положим \( n = 2^k \)):
\[
  \sum\limits_{i=1}^{\log_2 n} \frac{n^{i}}{2^{\frac{i(i-1)}2}} =
  \sum\limits_{i=1}^{k} 2^{ik + \frac i 2 - \frac{i^2} 2} = \sum
  2^{-(ai + bk)^2 + (ck +d)^2}
\]
\end{document}
