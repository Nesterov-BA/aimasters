\documentclass{article}
\usepackage[T2A]{fontenc}
\usepackage[utf8]{inputenc}
\usepackage[english, russian]{babel}

% Set page size and margins
% Replace `letterpaper' with`a4paper' for UK/EU standard size
\usepackage[a4paper,top=2cm,bottom=2cm,left=2cm,right=2cm,marginparwidth=1.75cm]{geometry}

\usepackage{amsmath}
\usepackage{graphicx}
\usepackage[colorlinks=true, allcolors=blue]{hyperref}
\usepackage{amsfonts}
\usepackage{amssymb}
% \usepackage[left=1cm,right=1cm,top=1cm,bottom=1cm]{geometry}
\usepackage{hyperref}
\usepackage{seqsplit}
\usepackage[dvipsnames]{xcolor}
\usepackage{enumitem}
\usepackage{algorithm}
\usepackage{algpseudocode}
\usepackage{algorithmicx}
\usepackage{mathalfa}
\usepackage{mathrsfs}
\usepackage{dsfont}
\usepackage{caption,subcaption}
\usepackage{wrapfig}
\usepackage[stable]{footmisc}
\usepackage{indentfirst}
\usepackage{rotating}
\usepackage{pdflscape}

\usepackage{minted}

\usepackage{MnSymbol,wasysym}

\begin{document}

\begin{center}
  \Large {Задание 1. Асимптотические сложности.}
\end{center}

\bigskip

\textbf{1.1.а)} Да, например при \( f(n) = n^2 \), \( g(n)
\frac{n}{\log{n}} \) \( h(n) \) будет равно \( n\log n \).

\textbf{1.1.б)} Из условия получаем следующие неравенства:
\[
f(n) \le c_1 n^2,
\]
\[
\frac 1 {g(n)} \le c_2.
\]
для каких-то положительных констант \( c_1, c_2 \). Отсюда
\begin{equation}
h(n) \le c_1 c_2 n^2, \label{eq:11b}
\end{equation}  а для условия \( h(n) = \Theta (n^3) \) должно
выполняться неравенство \( h(n) \ge c_3 n^3 \), которое не может
выполняться одновременно с неравенством (\ref{eq:11b}) при всех \( n
\) ни для каких \( c_1,c_2,c_3 > 0 \).
\medskip

\textbf{3.} Докажем оценку сверху:
\[
\sum\limits_{i=1}^n i^\alpha \le  \sum\limits_{i=1}^n n^\alpha =
n\cdot n^\alpha = n^{1+\alpha}.
\]
. Оценка снизу:
\[
\sum\limits_{i=1}^n i^\alpha \ge \sum\limits_{i=\frac n 2}^n i^\alpha
\ge \sum\limits_{i=\frac n 2}^n {\frac n 2}^\alpha = \left(\frac n
2\right)^{1+\alpha } =\left(\frac{1}{2}\right)^{1+\alpha }n^{1+\alpha }.
\]

\medskip

\textbf{2.} Воспользуемся результатами предыдущей задачи. Для этого заметим,
что \( i^3 \le i^3 + 2i + 5 \) при всех \( i \) и \( i^3 + 2i + 5 \le 2i^3 \)
начиная с некоторого \( i \). Отсюда
\[
\sum\limits_{i=1}^n i^{\frac 3 2} \le \sum\limits_{i=1}^n
\sqrt{i^3+2i+5} \le \sqrt 2 \sum\limits_{i=1}^n i^{\frac 3 2}.
\]
По задаче 3 каждая из сумм стоящих слева и справа равна \( \Theta
(n^{\frac 5 2}). \) Значит и \( \sum\limits_{i=1}^n \sqrt{i^3+2i+5} =
\Theta (n^{\frac 5 2}) \).

\textbf{Ответ:} \( \Theta (n^\frac 5 2). \)
\medskip

\textbf{4.} Положим \( f(x) = \frac 1 k \) при \( x \in [k,k+1) \) на
\( [1,\infty) \). Тогда \( g(n) = \int\limits_1^{n+1} f(x)dx = 1 +
\int\limits_2^{n+1} f(x)dx. \) При этом, на \( x\in[2,n+1] \)
\[
\frac 1 {x} \le f(x) \le \frac 1 {x-1}.
\]
Получаем \( 1 + \int\limits_2^{n+1} \frac 1 {x}  \le g(n) \le  1 +
\int\limits_2^{n+1} \frac 1 {x-1}\). Обе функции слева и справа
являются \( \Theta (\log n) \), значит \( g(n) = \Theta (\log n) \).

\textbf{Ответ:} \( \Theta (\log n). \)
\medskip

\end{document}
