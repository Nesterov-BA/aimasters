\documentclass{article}
\usepackage{myrussian}
\usepackage{MnSymbol,wasysym}

\def\labelenumi{\rm(\theenumi)}
\newcommand{\code}[1]{\texttt{#1}}

\begin{document}
\textbf{1.}
При любом переходе из комнаты в комнату будем
оставлять монеты в начале и в конце корридора.
Таким образом, при нахождении в комнате точно
знаем, в каких из соседних уже были.
\\
\textbf{Алгоритм:}
\begin{enumerate}
  \item Посещаем произвольную непосещенную комнату.
  \item Если нет непосещенных комнат, возвращаемся назад.
  \item Повторяем шаги 1,2 пока не посетим все комнаты.
\end{enumerate}
Это алгоритм поиска в глубину, он обходит все
вершины графа (в нашем случае комнаты).
\\
\textbf{Асимптотика:}
Каждый коридор посещается \(2\) раза, поэтому \(O(m)\).
\medskip

\textbf{2.}  Воспользуемся алгоритмом
покраски графа в \(2\) цвета с лекции.
\\ \textbf{Алгоритм.}  \\ Выберем
произвольную вершину, покрасим ее в
синий.
Затем красим в красный все вершины,
которые ей смежны.  Далее на
произвольном
шаге красим в цвет, отличный от цвета
вершины все непокрашенные смежные ей
вершины. Повторяем этот шаг для всех
смежных ей вершин. Алгоритм
заканчивается,
когда все вершины покрашены.  Таким
образом, каждая вершина будет покрашена
в
синий или красный.
\\ \textbf{Коррекстность.}  \\
\begin{theorem}
  В графе есть цикл нечетной длины
  \(\iff\) после окончания алгоритма
  в графе есть смежные вершины одного
  цвета.
\end{theorem}
\begin{proof}
  \(\Rightarrow\) Пусть соседние
  вершины имеют разные цвета.
  Сопоставим вершинам в цикле нечетной
  длины числа от \(1\) до
  \(2k+1\) по порядку. По
  предположению, если вершина с четным
  номером покрашена в синий, то вершина
  с нечетным --- красная. Но
  тогда смежные вершины \(1\) и
  \(2k+1\) красные --- противоречие.
  \\
  \(\Leftarrow\) По построению
  алгоритма, если вершины одного цвета,
  то между ними есть путь четной длины.
  Если две соседние вершины
  одного цвета, то есть путь четной
  длины их соединяющий. Вместе с
  ребром, которым они уже соединены
  получаем цикл нечетной длины.
\end{proof}
\textbf{Асимптотика.}
\\
Каждая вершина обрабатывается один раз,
при этом каждое ребро проверяется \(2\)
раза. \(\Theta\left(|V|+|E|\right)\)
\end{document}
