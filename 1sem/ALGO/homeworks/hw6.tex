\documentclass{article}
\usepackage[T2A]{fontenc}
\usepackage[utf8]{inputenc}
\usepackage[english, russian]{babel}

% Set page size and margins
% Replace `letterpaper' with`a4paper' for UK/EU standard size
\usepackage[a4paper,top=2cm,bottom=2cm,left=2cm,right=2cm,marginparwidth=1.75cm]{geometry}

\usepackage{amsmath,amssymb, amsthm}
\usepackage{graphicx}
\usepackage[colorlinks=true, allcolors=blue]{hyperref}
\usepackage{amsfonts}
\usepackage{amssymb}
% \usepackage[left=1cm,right=1cm,top=1cm,bottom=1cm]{geometry}
\usepackage{hyperref}
\usepackage{seqsplit}
\usepackage[dvipsnames]{xcolor}
\usepackage{enumitem}
\usepackage{algorithm}
\usepackage{algpseudocode}
\usepackage{algorithmicx}
\usepackage{microtype}
\usepackage{mathalfa}
\usepackage{mathrsfs}
\usepackage{dsfont}
\usepackage{caption,subcaption}
\usepackage{wrapfig}
\usepackage[stable]{footmisc}
\usepackage{indentfirst}
\usepackage{rotating}
\usepackage{pdflscape}
\usepackage{centernot}

% \usepackage{minted}

\usepackage{MnSymbol,wasysym}

\def\labelenumi{\rm(\theenumi)}
\def\divby{\,\vdots\,}
\def\ndivby{\,\not\vdots\,}

\begin{document}

\begin{center}
  \Large {Задание 6. Модульная арифметика и алгоритм Евклида.}
\end{center}
\textbf{1}
\begin{enumerate}
  \item Ищем решение уравнения \( ax + by = d \), получаем \( a_1,b_1
    \). Искомым решением будет пара \( ka_1, kb_1 \). \label{task1:1}
  \item Пусть дано уравнение в целых числах на \( a,b \):
    \[
      ax + by = h.
    \]
    Это уравнение разрешимо тогда и только тогда, когда \( h\divby \gcd(x,y) \).
    \begin{proof}
      Обозначим \( d = \gcd(x,y)\). Тогда для любых целых \( a,b \)
      выполняется \( ax \divby d, by \divby d \Rightarrow ax + by
      \divby d \). То есть не существует целых \( a,b \), таких, что
      \( ax+by \ndivby d\). Отсюда следует необходимость условия \( h
      \divby \gcd(x,y) \). Достаточность --- пункт \ref{task1:1}.
    \end{proof}
  \item Пусть \( (a_1,b_1), (a_2, b_2) \) --- частные решения. Тогда
    \( a_1 x + b_1 y = a_2 x + b_2 y = d \). Отсюда
    \[
      (a_1 - a_2)x +
      (b_1 - b_2)y = 0.
    \]
    Пусть теперь \( a_0 x + b_0 y = 0 \). Тогда \( (a_1 + a_0)x +
    (b_1 + b_0) y = d \). Итак, все решения уравнения отличаются от
    произвольного частного на решение однородного. Общее решение
    уравнения \( ax + by = 0 \) несложно найти. Обозначим \( e =
    \text{НОК} (x,y) \) и перепишем уравнение в виде \( ax = -by \).
    Видно, что и левая и правая части должны делиться на \( x \) и \(
    y \), т.е. должны быть кратны \( e \). Поэтому общее решение
    имеет вид \( \left(\frac e x t, - \frac e y t\right), t \in \mathbb{Z}\).

    Итак, \( a_1, b_1 \) --- частное решение, \( a_0 = \frac e x, b_0
    = -\frac e y\). Тогда общее решение исходного уравнения имеет вид
    \( (a_1 + ta_0, b_1 + t b_0), t \in \mathbb Z.\)

\end{enumerate}

\textbf{2} Решите уравнения в целых числах. Нужно найти все решения,
а не только частное.

\begin{enumerate}
  \item $238x + 385y = 133$,
  \item $143x + 121y = 52$.
\end{enumerate}
\begin{enumerate}
  \item Найдем частное решение расширенным алгоритмом Евклида:
    \begin{gather*}
      s_{0} = 1, t_{0} = 0\\
      s_{1} = 0, t_{1} = 1\\
      385 = 238 + 147, s_{2} = 1, t_{2} = -1\\
      238 = 147 + 91, s_{3} = 0 - 1 = -1, t_{3} = 2\\
      147 = 91 + 56, s_{4} = 1 + 1 = 2, t_{4} = -3\\
      91 = 56 + 35, s_{5} = -1-2 = -3, t_{5} = 5\\
      56 = 35 + 21, s_{6} = 5, t_{6} = -8\\
      35 = 21 + 14, s_{7} = -8, t_{7} = 13\\
      21 = 14 + 7, s_{8} = 13, t_{8} = -21\\
      14 = 2*7
    \end{gather*}
    Получаем, что $283*(-21) + 385*13 = 7$, откуда $283*(-399) +
    385*(247) = 133$.
    Общее решение однородного уравнения:$x = 55t, y = -34t, t \in \mathbb{Z}$.

    Общее решение имеет вид: \[x = 55t - 283, y = -34t + 247.\]
  \item Найдем частное решение расширенным алгоритмом Евклида:
    \begin{gather*}
      s_{0} = 1, t_{0} = 0\\
      s_{1} = 0, t_{1} = 1\\
      143 = 121 + 22, s_{2} = 1, t_{2} = -1\\
      121 = 22*5 + 11 , s_{3} = 0 - 5 = -5, t_{3} = 6\\
      22 = 11*2
    \end{gather*}
    Получаем, что $143*(-5) + 121*5 = 11$, а $52$ не делится на $11$.
    \textbf{Решений нет}.

\end{enumerate}
\medskip

\textbf{3} Решите сравнение $68x + 85 \equiv 0 (\mod 561)$ с помощью
расширенного алгоритма Евклида. Требуется найти все решения в вычетах.\\
Решения уравнения экивалентно решению Диовантового уравнения $68x + 561y = -85$.
\begin{gather*}
  s_{0} = 1, t_{0} = 0\\
  s_{1} = 0, t_{1} = 1\\
  561 = 8*68 + 17, s_{2} = 1, t_{2} = -8\\
  68 = 17*4
\end{gather*}
Получаем $68*(-8) + 561*1 = 17$, откуда $68*(40) + 561*(-5) = -85$.\\
Общее решение однородного имеет вид $x = 33t, y = -4t, t\in \mathbb{Z}$.\\
\textbf{Ответ:} $x \equiv 40 + 33t (\mod 561), t = 0, \ldots, 16$.
\medskip

\textbf{4} Найдите обратный остаток $7^{-1} (\mod 102)$.

Обозначим искомый остаток за $x$. Тогда $x$ является решением
уравнения $7x + 102y = 1$.
\begin{gather*}
  s_{0} = 1, t_{0} = 0\\
  s_{1} = 0, t_{1} = 1\\
  102 = 14*7 + 4, s_{2} = 1, t_{2} = -14\\
  7 = 1*4 + 3, s_{3} = -1, t_{3} = 15\\
  4 = 3+1, s_{4} = 2, t_{4} = -29
\end{gather*}
Получаем $102*2 + 7*(-29) = 1$.
\textbf{Ответ:} $7^{-1}\equiv -29 \equiv 73(\mod 102)$.
\medskip

\end{document}
